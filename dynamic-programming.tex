\documentclass[a4paper, 12pt]{article}

\usepackage{amsmath}
\usepackage{amssymb}
\usepackage{titlesec}

\setlength{\parindent}{0pt}
\setlength{\parskip}{1em}

\titlespacing*{\section}{0pt}{0pt}{0pt} % {left}{before}{after}
\titlespacing*{\subsection}{0pt}{0pt}{0pt}
\titlespacing*{\subsubsection}{0pt}{0pt}{0pt}

\begin{document}

\title{Dynamic Programming}
\author{LQR471814}
\maketitle

\section{Basic ideas}

Any dynamic programming problem has a certain objective, minimizing cost, maximizing profits, maximizing utility, etc... The function that describes this objective is known as the \emph{objective function}.

Information about the current situation required to make a correct decision is known as \emph{state}. (ex. To gauge how much to spend, you must know how much you have. Therefore wealth $W$ would be one of the state variables)

Variables chosen by the agent are called \emph{chosen variables}, these are decisions made based on \emph{state variables} which influence the next state (though next state is often also influenced by stuff outside the \emph{chosen variables}).

Dynamic programming describes the optimal chain of decisions by finding a rule that determines the most optimal decision given any possible of the state. (ex. If consumption only depends on wealth, you would want to find a function $c(W)$ which gives the optimal consumption for any degree of wealth) This is known as a \emph{policy function}.

\section{Formalization}

Suppose $x_{t}$ is the state at a time $t$, the initial state being $x_{0}$.

At any point the set of possible actions is defined as $a_{t} \in \Gamma(x_{t})$, where: 1) $a_{t}$ are the particular values for a set of control variables (variables that influence the next state) 2) $\Gamma(x_{t})$ is the set of actions that can be taken at state $x_{t}$.

State changes from $x$ to a new state $T(x, a)$ when action $a$ is taken.

The objective function that calculates the optimality of a given action $a$ on a state $x$ is $F(x, a)$.

Finally, we can define the \emph{value function} (the function that describes the best possible value of the objective as a function of state $x$) like follows.

\[
  V(x) = \underset{a \in \Gamma(x)}{\text{max}}\{F(x, a) + V(T(x, a))\}
\]

\emph{Interpretation:} The value function is the maximum value of the objective function (given the current $a$) added to the value function of the next state (given the current $a$) for all $a$ in $\Gamma(x)$.

Usually we are given $x_0$, $T$, $F$, and $\Gamma$ as part of the problem description and are tasked with finding $V(x)$ (the most optimal value possible for a given starting state) and $a(x)$ (the optimal decision to make at a given state).

\section{Examples}

\subsection{Optimal stopping problem}

Suppose a person is evaluating potential employment opportunities for the next 10 years ($t = 1, 2, 3, ...$).

At each value $t$, they may encounter a choice between a ``good'' job offering $\$100$ in salary or a ``bad'' job offering $\$44$, each with an equal probability of being offered. 

\begin{itemize}
  \item Pick the available job offered.
  \item Reject the offer and wait till the next year.
\end{itemize}

The question then becomes, what are the choices that should be made over the 10 year period to maximize money obtained?

This can be solved by reasoning backwards from $t=10$:

\begin{enumerate}
  \item At $t = 10$ the total earnings from accepting a ``good'' job is $\$100$; the value of accepting a ``bad'' job is $\$44$; the total earnings from rejecting the available job is $\$0$. Therefore, if they are still unemployed in the last period, they should accept whatever job they are offered at that time for greater income.
  \item At $t = 9$, the total earnings from accepting a ``good'' job is $2 \times 100 = 200$ because that job will last for two years. The total earnings from accepting a ``bad'' job is $2 \times 44 = 88$. The total expected earnings from rejecting a job offer are $0$ now plus the value of the next job offer, which will either be $44$ with $\frac{1}{2}$ probability or $100$ with $\frac{1}{2}$ probability, for an average (``expected'') value of $\frac {\$100+\$44}{2}=\$72$. Therefore, the job available at $t=9$ should be accepted.
  \item At $t=8$, the total earnings from accepting a ``good'' job is $3\times \$100=\$300$; the total earnings from accepting a ``bad'' job is $3\times \$44=\$132$. The total expected earnings from rejecting a job offer is $\$0$ now plus the total expected earnings from waiting for a job offer at $t=9$. As previously concluded, any offer at $t=9$ should be accepted and the expected value of doing so is ${\frac {\$200+\$88}{2}}=\$144$. Therefore, at $t=8$, total expected earnings are higher if the person waits for the next offer rather than accepting a ``bad'' job.
\end{enumerate}


By continuing to work backwards, it can be verified that a ``bad'' offer should only be accepted if the person is still unemployed at $t=9$ or $t=10$; a bad offer should be rejected at any time up to and including $t=8$. Generalizing this example intuitively, it corresponds to the principle that if one expects to work in a job for a long time, it is worth picking carefully.

\subsubsection{Math interpretation}

\[
\begin{aligned}
  x &= (o, j, t) \\
  x_{o} &\in \{44, 100\} \\
  x_{j} &\in \{\text{has job}, \text{no job}\} \\
  \{x_{t} &\in \mathbb{Z} | 1 \leq x \leq 10\}
\end{aligned}
\]

\begin{itemize}
  \item $x$ is the state.
  \item $x_{o}$ is the amount of money you would earn if you had a job.
  \item $x_{j}$ is your employment status.
  \item $x_{t}$ is the current time.
\end{itemize}

\[
\begin{aligned}
  a(x) \in \Gamma(x) = \begin{cases}
    \{\text{no change}\}, & x_{j} = \text{has job} \\
    \{\text{no change}, \text{take job}\}, & x_{j} = \text{no job}
  \end{cases}
\end{aligned}
\]

\begin{itemize}
  \item $a(x)$ is the optimal action to take for a given state $x$.
  \item $\Gamma(x)$ is the possible actions that can be taken for a given state $x$.
\end{itemize}

\[
\begin{aligned}
  T(x, a) = \begin{cases}
    (x_{o}, \text{has job}, x_{t} + 1), & a = \text{take job} \\
    \begin{cases}
      (\text{rand}(100, 44), x_{j}, x_{t}+1), & x_{j} = \text{no job} \\
      (x_{o}, x_{j}, x_{t}+1), & x_{j} = \text{has job}
    \end{cases}, & a = \text{no change}
  \end{cases}
\end{aligned}
\]

\begin{itemize}
  \item $T(x, a)$ defines the next state for a given current state $x$ and chosen action $a$.
\end{itemize}

\[
\begin{aligned}
  F(x, a) = \begin{cases}
    x_{o}, & a=\text{take job} \\
    \begin{cases}
      x_{o}, & x_{j}=\text{has job} \\
      0, & x_{j} = \text{no job}
    \end{cases}, & a=\text{no change}
  \end{cases}
\end{aligned}
\]

\begin{itemize}
  \item $F(x,a)$ is the objective function that determines how optimal taking the given action $a$ on the current state $x$ is.
\end{itemize}

\[
  V(x)=\underset{a \in \Gamma(x)}{\text{max}}\{F(x,a)+V(T(x,a))\}
\]

We can see that we are given $x$, $\Gamma(x)$, $T(x, a)$, and $F(x, a)$ from the problem. Our job now, is to determine what $V(x)$ and $a(x)$ are, most commonly through processes of induction.

\subsubsection{Solving for the equations}

We'll use the technique of backwards induction to derive the equations $V(x)$ and $a(x)$.

This involves starting from a base case, then branching off from that base case and seeing what the commonalities are.

\textbf{Base cases}

\[
\begin{aligned}
  V((100, \text{no job}, 10)) &= \text{max}\{100, 0\} \\
                              &= 100 \\
  a((100, \text{no job}, 10)) &= \text{take job} \\
  V((44, \text{no job}, 10)) &= \text{max}\{44, 0\} \\
                             &= 44 \\
  a((44, \text{no job}, 10)) &= \text{take job}
\end{aligned}
\]

Since the offer has a probability of being either $100$ or $44$, we compute the ``expected'' optimal value at year $10$ as the average between the optimal value at $x_{o}=100$ or $x_{o}=44$.

\[
\begin{aligned}
  V((\{100, 44\}, \text{no job}, 10)) &= \frac{100+44}{2} \\
                                                &= 72
\end{aligned}
\]

If you have a job at year $10$ then you just keep it (no other choices are available).

\[
\begin{aligned}
  V((\mathbb{R}, \text{has job}, 10)) &= \text{max}\{x_{o}\} \\
                                                &= x_{o} \\
  a((\mathbb{R}, \text{has job}, 10)) &= \text{no change}
\end{aligned}
\]

\textbf{Starting from year 9}

\[
\begin{aligned}
  V((\mathbb{R}, \text{has job}, 9)) &= F(x, \text{no change}) + V(T(x, \text{no change})) \\
                                &= x_{o} + V((x_{o}, \text{has job}, 10)) \\
                                &= x_{o} + x_{o} \\
                                &= 2x_{o}
\end{aligned}
\]

Like always, if you have a job already, you just keep it.

\[
\begin{aligned}
  V((100, \text{no job}, 9)) &= \text{max} \begin{cases}
    F(x, \text{take job}) + V(T(x, \text{take job})) \\
    F(x, \text{no change}) + V(T(x, \text{no change}))
  \end{cases} \\
                             &= \text{max} \{72, 200\} \\
                             &= 200 \\
  V((44, \text{no job}, 9)) &= \text{max} \begin{cases}
    F(x, \text{take job}) + V(T(x, \text{take job})) \\
    F(x, \text{no change}) + V(T(x, \text{no change}))
  \end{cases} \\
                            &= \text{max} \{72, 88\} \\
                            &= 88 \\
  a((\{44, 100\}, \text{no job}, 9)) &= \text{take job}
\end{aligned}
\]

And the expected optimal value at year 9.

\[
  V((\{44, 100\}, \text{no job}, 9)) = 144
\]

\textbf{Starting from year 8}

\[
\begin{aligned}
  V((\mathbb{R}, \text{has job}, 8)) &= F(x, \text{no change}) + V(T(x, \text{no change})) \\
                                &= 3x_{o} \\
  V((100, \text{no job}, 8)) &= \text{max} \begin{cases}
    F(x, \text{take job}) + V(T(x, \text{take job})) \\
    F(x, \text{no change}) + V(T(x, \text{no change}))
  \end{cases} \\
                             &= \text{max}\{144, 300\} \\
                             &= 300 \\
  V((44, \text{no job}, 8)) &= \text{max} \begin{cases}
    F(x, \text{take job}) + V(T(x, \text{take job})) \\
    F(x, \text{no change}) + V(T(x, \text{no change}))
  \end{cases} \\
                             &= \text{max}\{144, 132\} \\
                             &= 144 \\
    a((100, \text{no job}, 8)) &= \text{take job} \\
    a((44, \text{no job}, 8)) &= \text{no change} \\
    V((\{44, 100\}, \text{no job}, 8)) &= \frac{300+144}{2} \\
                                       &= 222
\end{aligned}
\]

If you go through the remaining years and use some inductive reasoning, you can see that the optimal strategy to maximize salary over this 10 year period is:

\[
  a(x) = \begin{cases}
    \text{no change}, & x_{j} = \text{has job} \\
    \begin{cases}
      \text{take job}, & x_{t} \geq 9 \\
      \begin{cases}
        \text{take job}, & x_{o} = 100 \\
        \text{no change}, & x_{o} = 44
      \end{cases}, & x_{t} < 9
    \end{cases}, & x_{j} = \text{no job}
  \end{cases}
\]

\end{document}
